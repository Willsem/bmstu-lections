\documentclass[a4paper, 14pt]{extreport}

\usepackage[utf8]{inputenc}
\usepackage[T2A]{fontenc}
\usepackage[english,russian]{babel}
\usepackage{listings}
\usepackage[table]{xcolor}
\usepackage{amsmath}
\usepackage{amsfonts}
\usepackage{MnSymbol}
\usepackage{wasysym}
\usepackage{indentfirst}
\usepackage{hyperref}
\usepackage{nomencl}
\usepackage{pgfplots}
\usepackage{geometry}
\usepackage{graphicx}
\usepackage{float}
\usepackage{multirow}
\usepackage{setspace}
\usepackage{blindtext}
\usepackage{titlesec, blindtext, color}

\hypersetup{
    bookmarks=true,
    colorlinks=true,
    unicode=true,
    urlcolor=black,
    linkcolor=black,
    anchorcolor=black,
    citecolor=black,
    menucolor=black,
    filecolor=black
}

\pgfplotsset{compat=1.9}

\geometry{top=20mm}
\geometry{right=15mm}
\geometry{left=15mm}
\geometry{bottom=20mm}
\geometry{ignorefoot}

\onehalfspacing
\setlength{\parskip}{1ex}
\renewcommand\labelitemi{---}

\lstset{
    language=c,
    numbers=left,
    texcl=true,
    basicstyle=\small,
    escapebegin=\begin{russian}\commentfont,
    escapeend=\end{russian},
    literate={Ö}{{\"O}}1
    {Ä}{{\"A}}1
    {Ü}{{\"U}}1
    {ß}{{\ss}}1
    {ü}{{\"u}}1
    {ä}{{\"a}}1
    {ö}{{\"o}}1
    {~}{{\textasciitilde}}1
    {а}{{\selectfont\char224}}1
    {б}{{\selectfont\char225}}1
    {в}{{\selectfont\char226}}1
    {г}{{\selectfont\char227}}1
    {д}{{\selectfont\char228}}1
    {е}{{\selectfont\char229}}1
    {ё}{{\"e}}1
    {ж}{{\selectfont\char230}}1
    {з}{{\selectfont\char231}}1
    {и}{{\selectfont\char232}}1
    {й}{{\selectfont\char233}}1
    {к}{{\selectfont\char234}}1
    {л}{{\selectfont\char235}}1
    {м}{{\selectfont\char236}}1
    {н}{{\selectfont\char237}}1
    {о}{{\selectfont\char238}}1
    {п}{{\selectfont\char239}}1
    {р}{{\selectfont\char240}}1
    {с}{{\selectfont\char241}}1
    {т}{{\selectfont\char242}}1
    {у}{{\selectfont\char243}}1
    {ф}{{\selectfont\char244}}1
    {х}{{\selectfont\char245}}1
    {ц}{{\selectfont\char246}}1
    {ч}{{\selectfont\char247}}1
    {ш}{{\selectfont\char248}}1
    {щ}{{\selectfont\char249}}1
    {ъ}{{\selectfont\char250}}1
    {ы}{{\selectfont\char251}}1
    {ь}{{\selectfont\char252}}1
    {э}{{\selectfont\char253}}1
    {ю}{{\selectfont\char254}}1
    {я}{{\selectfont\char255}}1
    {А}{{\selectfont\char192}}1
    {Б}{{\selectfont\char193}}1
    {В}{{\selectfont\char194}}1
    {Г}{{\selectfont\char195}}1
    {Д}{{\selectfont\char196}}1
    {Е}{{\selectfont\char197}}1
    {Ё}{{\"E}}1
    {Ж}{{\selectfont\char198}}1
    {З}{{\selectfont\char199}}1
    {И}{{\selectfont\char200}}1
    {Й}{{\selectfont\char201}}1
    {К}{{\selectfont\char202}}1
    {Л}{{\selectfont\char203}}1
    {М}{{\selectfont\char204}}1
    {Н}{{\selectfont\char205}}1
    {О}{{\selectfont\char206}}1
    {П}{{\selectfont\char207}}1
    {Р}{{\selectfont\char208}}1
    {С}{{\selectfont\char209}}1
    {Т}{{\selectfont\char210}}1
    {У}{{\selectfont\char211}}1
    {Ф}{{\selectfont\char212}}1
    {Х}{{\selectfont\char213}}1
    {Ц}{{\selectfont\char214}}1
    {Ч}{{\selectfont\char215}}1
    {Ш}{{\selectfont\char216}}1
    {Щ}{{\selectfont\char217}}1
    {Ъ}{{\selectfont\char218}}1
    {Ы}{{\selectfont\char219}}1
    {Ь}{{\selectfont\char220}}1
    {Э}{{\selectfont\char221}}1
    {Ю}{{\selectfont\char222}}1
    {Я}{{\selectfont\char223}}1
    {і}{{\selectfont\char105}}1
    {ї}{{\selectfont\char168}}1
    {є}{{\selectfont\char185}}1
    {ґ}{{\selectfont\char160}}1
    {І}{{\selectfont\char73}}1
    {Ї}{{\selectfont\char136}}1
    {Є}{{\selectfont\char153}}1
    {Ґ}{{\selectfont\char128}}1
}

\titleformat
{\section} % command
{\bfseries\Large} % format
{§\thesection}
{0.2cm}
{
    \centering
}

\titleformat
{\subsection} % command
{\bfseries\large} % format
{\thesubsection}
{0.2cm}
{
    \centering
}

\titlespacing*{\section}{0cm}{0cm}{0cm}[0cm]

\newcommand{\anonsection}[1]{\section*{#1}\addcontentsline{toc}{section}{#1}}



\begin{document}

\titleformat
{\section} % command
{\bfseries\Large} % format
{§\thesection}
{0.2cm}
{
    \centering
}

\titleformat
{\subsection} % command
{\bfseries\large} % format
{\thesubsection}
{0.2cm}
{
    \centering
}

\titlespacing*{\section}{0cm}{0cm}{0cm}[0cm]

\newcommand{\anonsection}[1]{\section*{#1}\addcontentsline{toc}{section}{#1}}


\section{Курсовой проект}

\subsection{Направления}

\subsubsection{Желательно}

\begin{itemize}
    \item Разработка сетевого протокола для обмена данными между различными устройствами для решения сетевой задачи
    \item Система онлайн-трансляций (аудио- видео- конференцсвязь)
    \item Разработка драйвера для сетевой ОС для шифрования набора данных
    \item Система мониторинга сетевых сертификатов
    \item Приложение для удаленного управления устройством
    \item Система электронных транзакций (с поддержкой крипторграфических протоколов)
\end{itemize}

\subsubsection{Нежелательно}

\begin{itemize}
    \item Портал (сайт)
    \item Сетевое игровое приложение (часто реализуемые)
    \item Приложения для обмена файлами
    \item Чат
\end{itemize}

\subsection{Инструментарий}

\begin{enumerate}
    \item Среда разработки (для первых лр желательно C/C++)
    \item Виртуальный стенд
        \begin{itemize}
            \item Cisco Packet Tracer
            \item GNS
        \end{itemize}
    \item Wireshark
\end{enumerate}

\subsection{Литература}

\begin{itemize}
    \item Таненбаум <<Компьютерные сети>>
    \item Олифер, Олифер <<Компьютерные сети, принцип, технологии, протоколы>>
    \item Куроуз <<Компьютерные сети, нисходящие подходы>>
    \item Bees's Guide to Network Programming
    \item docs.microsoft.com
\end{itemize}

\subsection{Курсовая работа}

\begin{enumerate}
    \item Мат. модель
\end{enumerate}

\section{Классификация и обзор сетей}

\textbf{Компьютерная сеть} -- совокупность компьютеров и других устройств, соединенных линиями связи и обменивающихся информацией между собой в соответствии с определенными правилами -- протоколом.

\textbf{Протоколы} -- набор соглашений интерфейса логического уровня, которые определят обмен данными между разлиными программами. Эти соглашения задают единообразный способ передачи ошибок

\textbf{Сервер} -- специализированный компьютер и/или специализированное оборудование.



\end{document}
