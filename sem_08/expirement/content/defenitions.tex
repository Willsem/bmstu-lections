\section{Определения}

\textbf{Эксперимент} -- система наблюдений, воздействия, операций, направленных на получение информации об объекте при исследовательских испытаниях.

\textbf{Опыт} -- это воспроизведение поведения исследуюемого явления в определенных условиях проведения эксперимента при возможности регистрации его результатов.

\textbf{План эксперимента} -- это совокупность данных, определяющих число, условие и порядок проведения (реализации) опыта.

\textbf{Планирование эксперимента} -- выбор плана эксперимента, удовлетворяющее заданным требованиям.

\textbf{Фактор} -- это переменная, которая по предположению влияет на результат эксперимента.

\textbf{Уровень фактора} -- это фиксированное значение фактора относительно начало отсчета (безразмерная величина).

\textbf{Основной уровень фактора} -- натуральное значение фактора, которое соответствует нулевому уровню безразмерной величины.

\textbf{Нормализация фактора} -- преобразование натуральных величин в безразмерные величины.

\textbf{Априорное ранжирование факторов} -- метод выбора наиболее важных факторов, основанных на предварительном знании (экспертной оценке).

\textbf{Размах варьирования фактора} -- разность максимального и минимаьного значения факторов в натуральной величине.

\begin{equation*}
    \Delta I = I_{\max} - I_{\min}
\end{equation*}

\textbf{Интервал варьирования фактора} -- половина размаха варьирования фактора.

\begin{equation*}
    \frac{\Delta I}{2} = \frac{I_{\max} - I_{\min}}{2}
\end{equation*}

\textbf{Эффект взаимодействия факторов} -- показатель зависимости изменения эффекта одного фактора от уровня других факторов.

\textbf{Факторное пространство} -- пространство, координатной оси которой совпадают с факторами

\textbf{Область экспериментирования (планирования)} -- область факторного пространства, в которой выбираются точки, соответствующие условиям проведения эксперимента.

\textbf{Пассивный эксперимент} -- человек при проведении пассивного эксперимента не задает уровни факторов, а лишь регистрирует их значения.

\textbf{Активный эксперимент} -- человек при проведении активного эксперимента сам задает определенные значения факторов.

\textbf{Последовательный эксперимент} -- эксперимент, реализуемый в виде серии опытов, причем условие проведения каждой последующей серии определяется результатом предыдущей.

\textbf{Отклик} -- наблюдаемая случайная величина, по определению зависящая от фактора.

\textbf{Функция отклика} -- зависимость математического ожидания отклика от фактора.

\textbf{Оценка функции отклика} -- значение получаемое при подстановке в функцию отклика значения фактора.

\textbf{Дисперсия оценки функции отклика} -- дисперсия оценки математического ожидания.

\textbf{Поверхность функции отклика} -- геометрическое представление функции отклика.

\textbf{Область оптиума} -- область факторнго пространства в окрестности точки, в которой функция отклика достигает экстремального значения.

\textbf{Рандомизация плана} -- один из приемов планирования эксперимента, при котором влияние некоторой случайной величины сводят к случайной ошибке.

\textbf{Параллельные опыты} -- рандомизированные опыты, в которых значение всех факторов остаются неизменными.

При планировании эксперимента исследователь должен:

\begin{itemize}
    \item помнить к какому классу систем относится рассматриваемая система
    \item определять режим работы системы
\end{itemize}
