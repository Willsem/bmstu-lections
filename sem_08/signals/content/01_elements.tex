\section{Элементы теории сигналов}

\textbf{Сигнал} -- под сигналом понимается физический процесс отображающий сообщения и служащий для его передачи по каналу связи.

\subsection{Классификация}

\subsubsection{Критерии классификации}

\begin{itemize}
    \item множество значений, которые может принимать сигнал
    \item множество значений, которые принимают аргументы этого сигнала
\end{itemize}

В общем случае сигнал описывается функцией

\begin{equation*}
    U(x, y, z, t)
\end{equation*}

\begin{enumerate}
    \item Пространственный и временной
    \item Финитный и инфинитный
    \item Аналоговый и цифровой:

        \begin{itemize}
            \item Дискретный (аргументы не являются непрерывными, последовательность значений)
            \item Квантованный (аргументы конечные и дискретные)
        \end{itemize}

    \item Детерминированный и случайный
\end{enumerate}

\subsection{Математичское представление сигналов}

\begin{equation*}
    U = U_{R_e} + i U_{im}, i \text{ -- мнимая единица}
\end{equation*}

\subsubsection{Формула Эйлера}

\begin{equation*}
    U = \underbrace{U_0}_{\text{амплитуда}} e^{i \underbrace{\varphi}_{\text{фаза}}}
\end{equation*}

\begin{equation*}
    |U|^2 \text{ -- интенсивность}
\end{equation*}

\subsection{Основные свойства}

\begin{enumerate}
    \item Степень отличия 2-х сигналов

        \begin{itemize}
            \item среднеквадратичное отклонение

                \begin{equation*}
                    d =  \sqrt{\sum_{i=0}^{N-1}  \mid U_1(x_i) - U_2(x_i) \mid^2}
                \end{equation*}

            \item максимальное отклонение

                \begin{equation*}
                    d = \max_{i=0,1, \ldots , N-1} \mid U_1(x_i) - U_2(x_i) \mid
                \end{equation*}

            \item PSNR -- пиковое отношение <<сигнал/шум>>

                Для изображений:

                \begin{equation*}
                    d = \lg \frac{255^2 N^2}{\sum_{i,j=0}^{N-1} \mid U_{1_{ij}} - U_{2_{ij}} \mid^2}
                \end{equation*}

            \item визуальный критерий
        \end{itemize}

    \item Принцип суперпозиции -- результат действия двух или более сигналов равен их геометрической сумме

        \begin{equation*}
            U = U_1 + U_2
        \end{equation*}

        \begin{equation*}
            U \ne \mid U_1 \mid^2 + \mid U_2 \mid^2
        \end{equation*}

    \item Разложение по базисным функциям

        \begin{equation*}
            U = \sum_{k=0}^{\infty} U_k \varphi_k
        \end{equation*}

        $\varphi_k$ -- базисные функции

        $U_k$ -- коэффициент разложения
\end{enumerate}

\subsection{Дискретиация сигналов}

\textbf{Дискретизация сигналов} -- это замена непрерывного сигнала последовательностью чисел, называемых отсчетами, являющийся представлением этого сигнала по некоторому базису.

\subsubsection{Теорема Котельникова}

Сигналы, сперкт Фурье которых равен нулю за пределами интервала $(-F; F)$, могут быть точно восстановлены по своим отсчетам взятым с шагом $\Delta t = \frac{1}{2F}$ по следующей формуле

\begin{equation*}
    U(t) = \sum_{k=-\infty}^{+\infty} U(k \Delta t) \text{sinc} \bigg( 2 \pi F \big(t - \frac{k}{2F}\big) \bigg)
\end{equation*}

$\text{sinc} (x) = \frac{\sin(x)}{x}$ -- функция отсчета

\subsubsection{Спектр Фурье}

\begin{equation*}
    U(t) : V(f) = \int_{-\infty}^{+\infty} u(t) \exp (-2\pi ift) dt \text{ -- преобразования Фурье}
\end{equation*}

$f$ -- частота
