\usepackage[utf8]{inputenc}
\usepackage[T2A]{fontenc}
\usepackage[english,russian]{babel}
\usepackage{listings}

\usepackage[table]{xcolor}

\usepackage{amsmath}
\usepackage{MnSymbol}
\usepackage{wasysym}
\usepackage{indentfirst}
\usepackage[
bookmarks=true, colorlinks=true, unicode=true,
urlcolor=black,linkcolor=black, anchorcolor=black,
citecolor=black, menucolor=black, filecolor=black,
]{hyperref}
\usepackage{nomencl}

\usepackage{pgfplots}
\pgfplotsset{compat=1.9}

\usepackage{geometry}
\geometry{top=15mm}
\geometry{right=15mm}
\geometry{left=15mm}
\geometry{bottom=20mm}
\geometry{ignorefoot}

\usepackage{graphicx}

\usepackage{float}

\usepackage{multirow}

\usepackage{setspace}
\onehalfspacing

\setlength{\parskip}{1ex} % разрыв между абзацами
\usepackage{blindtext}

\usepackage{titlesec, blindtext, color}

\definecolor{gray75}{gray}{0.75}
\newcommand{\hsp}{\hspace{20pt}}
\titleformat{\chapter}[hang]{\Huge\bfseries}{\thechapter\hsp\textcolor{gray75}{|}\hsp}{0pt}{\Huge\bfseries}

\usepackage{mdframed}
\newmdtheoremenv{theorem}{Теорема}[section]
\newtheorem{lemma}{Лемма}[section]
\newtheorem{defenition}{Определение}[section]
\newtheorem{note}{Замечание}[section]
\newtheorem{cont}{Следствие}[section]
\newtheorem{example}{Пример}[section]
