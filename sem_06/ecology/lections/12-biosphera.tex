\chapter{Биосфера. Учение Вернандского}

\section{4 типа вещества}

\begin{enumerate}
    \item \textbf{Косное вещество} --
        биологическое вещество не связанное с деятельностью
        живых организмов

    \item \textbf{Живое вещество} --
        все живые организмы

    \item \textbf{Биогенное вещество} --
        связаны с жизнедеятельностью живых организмов

    \item \textbf{Биокосное вещество} --
        тесная связь между биогенным и косным веществом
\end{enumerate}

\section{Функции живого вещества}

\begin{enumerate}
    \item \textbf{Концентрационная функция} --
        организмы концентрируют в себе химические элементы

    \item \textbf{Газовая функция} --
        связана с текущим составом атмосферы

    \item \textbf{Окислительно-восстановительная функция} --
        способствование химическим реакциям

    \item \textbf{Био-химическая функция} --
        распространение живым веществом химических элементов по планете

        \textbf{Закон биогенной миграции атомов} --
        миграция химических элементов на земной поверхности
        и в биосфере в целом осуществляется или при непостредственном
        участии живого вещества, или же она протекает в среде,
        геохиимческие особенности которой обусловлены живым веществом
        как тем, которым настоящее время населяют биосферу, так и тем,
        которые действовали на Земле в течение всей ее геологической
        истории.

    \item \textbf{Биохимическая деятельность человека} --
        связана с действием человека на планете

    \item \textbf{Информационная функция} --
        ДНК и РНК, в которых накапливается информация

    \item \textbf{Средообразующая и средорегулирующая функции}

    \item \textbf{Энергетическая функция} --
        связана как с накоплением энергии в живом организме и передачи ее
        другим организмам в пищевой цепочке
\end{enumerate}

\section{Границы атмосферы}

\begin{itemize}
    \item Озоновый слой
    \item Водная среда -- вся толща до марианской впадины
    \item Суша -- до глубины, где 100 C
\end{itemize}

\section{Условия необходимые для становления ноосферой}

\textbf{Ноосфера} -- сфера разума (1927г. Э.Леруа)

\begin{enumerate}
    \item Заселение человеком всей планеты
    \item Резкое преобразование средств связи и обмена между странами
    \item Начало преобладания геологической роли человека над
        геологическими процессами, происходящими в биосфере
    \item Усиление связей (в том числе политических) между странами
    \item Расширение границ биосферы и выход в космос
    \item Открытие новых источников энергии
    \item Равенство людей, всех рас и религий
    \item Увеличение ролей народных масс в решении вопросов
        внешней и внутренней политики
    \item Свобода научной мысли, научного искания
    \item Продуманная система народного образования, подъем
        благосостояния трудящихся, создание реальной
        возможности не допустить голода и
        нищеты и чрезвычайно ослабить болезни
    \item Разумное преобразование первичной природы Земли с целью сделать
        ее способной, удовлетворить все материальные эстетические и
        материальные потребности численно возрастающего населения
    \item Исключение вольной жизни в обществе
\end{enumerate}
