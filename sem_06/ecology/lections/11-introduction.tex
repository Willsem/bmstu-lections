\chapter{Введение}

\section{Категории отходов (неприродные)}

\begin{itemize}
    \item Радиоактивные отходы
    \item Медицинские
    \item Химические
\end{itemize}

\section{Ученые}

\begin{enumerate}
    \item \textbf{Ламарк} -- Выдвинул первые представления о биосфере
    \item В 1848 \textbf{Ферхюльст} вывел логичстические
        уравнения скорости роста численности

        \begin{equation*}
            \frac{dP}{dt} = r \cdot P,
            r - \text{ константа}
        \end{equation*}

        \begin{equation*}
            P = P_0 e^{rt}
        \end{equation*}

        \begin{equation*}
            \frac{dP}{dt} = r \cdot P \bigg( 1 - \frac{P}{k} \bigg),
            k - \text{ емкость экологической ниши популяции}
        \end{equation*}

        \begin{equation*}
            \frac{P}{k-P} = e^t \frac{P_0}{k - P_0}
        \end{equation*}

        \begin{equation*}
            P = \frac{k \cdot P_0 e^{vt}}{k + P_0 (e^{vt} - 1)}
        \end{equation*}

        Численность популяции стремится  к $k$ количеству, которое
        можно прокормить.

    \item Первая половина 19 века. \textbf{Рулье} изучал животных и
        выдвинул ряд законов

    \item Немецкий биолог \textbf{Геккель} в 1866 году ввел
        термин экология

        \textbf{Экология} -- это наука, изучающая живые организмы,
        среду их обитания и их взаимосвязи.

    \item Австралиец \textbf{Зюс} в 1875 ввел термин \textbf{биосфера}
        -- живая оболочка Земли.

    \item 1927 год \textbf{Элтон} выпустил первый учебник
        монографии по экологии

    \item Ученый-эколог \textbf{Коммонер} в 1974 выдвинул
        4 закона экологии

        \begin{enumerate}
            \item Все связано со всем
            \item За все надо платить (Нично не дается даром)
            \item Все должно куда-то деваться
            \item Природа знает лучше
        \end{enumerate}

        \textbf{ДДТ} -- стойкое органичесое вещество, вредное и опасное.
        Против комаров.
\end{enumerate}

\section{Три общетеоретические задачи экологии}

\begin{enumerate}
    \item Изучение биоразнообразия и способов его поддержания
    \item Разработка общей теории устойчивости экосистем
    \item Изучение экологических механизмов адаптации
        (приспособления) к среде
\end{enumerate}

\section{Три прикладные задачи экологии}

\begin{enumerate}
    \item Прогнозирование и оценка возможных отрицательных последствий
        для окружающей природной среды от хозяйственной
        деятельности человека

    \item Сохранение воспроизводства и рациональное
        использование природных ресурсов

    \item Оптимизация любых решений с точки зрения обеспечения
        экологически-безопасного устойчивого развития
\end{enumerate}

\section{Признаки живого}

\begin{enumerate}
    \item Клеточное строение
    \item Химический состав (C, H, O, N)
    \item Способность к росту и развитию
    \item Способность к размножению и наследственность
    \item Постонянный обмен веществом
    \item Способность к адаптации
    \item Наличие ответной реакции (раздражимость)
    \item Изменчивость
\end{enumerate}

\section{Уровни организации живой материи}

\begin{itemize}
    \item Молекулярный уровень
    \item Клеточный уровень
    \item Тканевый уровень
    \item Организменный уровень
    \item Популяционный (видовой) уровень
    \item Экосистемный уровень

        \textbf{Экосистема} -- это совокупность живых организмов
        и среды их обитания, образующее единое функциональное целое.

    \item Биосферный (глобальный) уровень
\end{itemize}
