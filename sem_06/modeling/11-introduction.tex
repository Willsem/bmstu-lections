\chapter{Общие понятия. Вводные замечания}

\textbf{Моделирование} -- это методология разработки
и изучения объекта, основанная на замене этого объекта моделью
и работе в дальнейшем с этой моделью.
Под объектом понимается система, процесс, явление, собственно
объект (материальная или нематериальная сущность).

\textbf{Модель} -- представление объекта в виде, отличном от облика
или формы его реального существования или функционирования
(упрощение).

\section{Классификация моделей}

Все модели можно разделить на три группы

\begin{enumerate}
    \item \textbf{Натурные (материальные)} --
        это модели, которые воспроизводят процессы в натуральном виде

        \begin{itemize}
            \item физические модели;
            \item геометрические модели;
            \item аналоговые модели и др.
        \end{itemize}

    \item \textbf{Абстрактные (идеальные)}

        \begin{itemize}
            \item символьные модели;
            \item интуитивные модели и др.
        \end{itemize}

    \item \textbf{Модели суждения (отношение к реалиям)}
\end{enumerate}

\textbf{Математическая модель} -- это представление объекта
в виде уравнений, логических соотношений, формул. Математическое
моделирование -- это методология изучения объекта, путем замены
его математической моделью.

\section{Классификация математических моделей}

\begin{itemize}
    \item \textbf{Иммитационные} --
        это модели типа массового обслуживания
    \item \textbf{Функциональные (регулярные)}
    \item \textbf{Модели идентификации}
\end{itemize}

\textbf{Предмет курса} -- изучение эффективных алгоритмов решения
вычислительных задач, возникающих при исследовании математических
моделей.

\section{Требования к моделям}

\begin{enumerate}
    \item Адекватность
        (соответствие процессу или объекту и его требованиям)
    \item Универсальность
        (описание не одного процесса, а группы)
    \item Точность
        (модель должна удовлетворять требованиям точности результата)
    \item Эконосичность
        (соотношение цена/качество)
\end{enumerate}

\section{Области деятельности, в которых оправданно мат. моделирование}

\begin{enumerate}
    \item Прогнозирование событий
    \item Исследование объектов, которые при каждом своем
        взаимодействии уничтожаются
    \item Дорогостоющие объекты с длительными сроками разработки
    \item Отсутствие объекта-оригинала
    \item Исследование длительной по времени эволюции процесса
\end{enumerate}

\section{Лабораторная работа 1}

Тема: Приближенный аналитический метод Пикара в сравнении с численными
методами

\begin{equation*}
    \begin{cases}
        u'(x) = f(x,u) \\
        u(\xi) = y
    \end{cases}
\end{equation*}

\begin{equation*}
    y^{(s)}(x) = \eta + \int_0^x f(t, y^{(s-1)}(t)) dt
\end{equation*}

\begin{equation*}
    y^{(0)} = \eta
\end{equation*}

\begin{equation*}
    \begin{cases}
        u'(x) = x^2 + u^2 \\
        u(0) = 0 \\
    \end{cases}
\end{equation*}

\begin{equation*}
    y^{(1)} = 0 + \int_0^x t^2 dt = \frac{x^3}{3}
\end{equation*}

\begin{equation*}
    y^{(2)} = 0 + \int_0^x \bigg[ t^2 + \big( \frac{t^3}{3} \big)^2 \bigg] dt = \frac{t^3}{3} + \frac{t^7}{7 \cdot 9} = \frac{t^3}{3} \bigg[1 + \frac{t^4}{21} \bigg]
\end{equation*}

\subsection{Численный метод}

Отрезок разбивается с каким-то шагом

\subsubsection{Явная схема}

\begin{equation*}
    \frac{y_{n+1} - y_n}{h} = f(x_n, y_n)
\end{equation*}

\begin{equation*}
    y_{n+1} = y_n + h f(x_n, y_n)
\end{equation*}

$h$ -- шаг

\begin{equation*}
    f(x_n, y_n) = x_n^2 + y_n^2
\end{equation*}

\subsubsection{Неявная схема}

\begin{equation*}
    y_{n+1} = y_n + h (f(x_{n+1}, y_{n+1}))
\end{equation*}

\begin{equation*}
    y_{n+1} = y_n + h \cdot (x^2_{n+1} + y^2_{n+1}) -
    \text{квадратное уравнение}
\end{equation*}

\begin{table}[H]
    \centering
    \begin{tabular}{|c|c|c|c|}
        \hline
        x & Пикар (любое приближение) & Явная схема & Неявная схема \\
        \hline
          & & & \\
    \end{tabular}
\end{table}

До какого $x$ вы добрались

\textbf{Тема для отчета:}
ОДУ. Задача Коши. Приближенный метод Пикара. Численный метод Эйлера.

\textbf{Цель работы:}
Изучить методы решения задачи Коши для ОДУ, применив
приближенный аналитический метод Пикара и численный метод Эйлера
в явном и неявном вариантах.

\textbf{Задание:}
Решить уравнение, не имеющее аналитического решения
