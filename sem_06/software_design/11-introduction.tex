\chapter{Введение}

\section{Цель и задачи}

\subsection{Цель}

Ответить для себя на вопрос <<Что такое хорошая Архитектура?>>

\subsection{Задачи}

\begin{itemize}
    \item Критерии хорошей Архитектуры
    \item Проектирование программных компонентов (SOLID)
    \item Проектирование программных систем
    \item Жизненный цикл разработки ПО
    \item Организация процесса разработки
    \item Организация процесса развертывания и сопровождения
\end{itemize}

\section{Архитектура}

\begin{itemize}
    \item Сфера применения
    \item Модификация
    \item Поддержка
\end{itemize}

\subsection{Архитектура программного обеспечения}

\textbf{Архитектура программного обеспечения} -- совокупность
важнейших решений об организации программной системы.

\begin{itemize}
    \item выбор структурных элементов и их интерфейсов,
        с помощью которых составлена система, а также
        их поведения в рамках сотрудничества структурных
        элементов
    \item соединение выбранных элементов структуры и поведения
        во всё более крупные системы
    \item общий архитектурный стиль
\end{itemize}

\subsection{Гради Буч (создатель UML)}

Архитектура отражает важные проектные решения по формированию системы,
где важность определяется стоимостью изменений

\subsection{Джойзеф Йодер}

Если вы думаете, что хорошая архитектура стоит дорого, попробуйте
плохую архитектуру

\subsection{Ральф Джонсон}

Архитектура -- это набор верных решений, которые хотелось бы принять
на ранних этапах работы над проектом, но которые не более вероятны,
чем другие

\subsection{Том Гилб}

Архитектура -- это гипотеза, которую требуется доказать
реализацией и оценкой

\subsection{Роберт Мартин}

Поспешай не торопясь

\textbf{Поспешность} -- самонадеянность, управляющая
перепроектированием, приведет к тому же беспорядку что и прежде

\subsection{Программные архитектуры}

\begin{itemize}
    \item Неизменны на всем протяжении развития IT, начиная с
        50-60х гг XX века
    \item Низкоуровневые детали и высокоуровневая структура являются
        частями одного целого
    \item Цель -- уменьшение трудозатрат на создание
        и сопровождение системы
\end{itemize}

\textbf{Идея ПО} -- простая возможность менять поведение компьютера

\subsection{Битва за архитектуру}

\begin{itemize}
    \item Разработчики -- архитектура
    \item Менеджмент -- функциональность
\end{itemize}

\section{Парадигмы программирования}

\begin{itemize}
    \item Структурное программирование
    \item Объектно-ориентированное программирование
    \item Функциональное программирование
\end{itemize}

\subsection{Структурное программирование}

1968, принципы Дейкстры

Структурное программирование накладывает ограничение на прямую
передачу управления

\begin{enumerate}
    \item Декомпозиция на подпрограммы
    \item Одна точка входа и выхода во всех конструкциях
    \item Проектирование сверху вниз
    \item Запрет безусловного перехода
    \item Любой алгоритм кодируется тремя структурами:
        последовательность, ветвление и цикл
    \item Деление на блоки
    \item Базовые конструкции могут быть вложены друг в друга
\end{enumerate}

\subsection{Объектно-ориентированное программирование}

Объектно-ориентированное программирование накладывает ограничение
на косвенную передачу управления

\begin{itemize}
    \item Инкапсуляция
    \item Наследование
    \item Полиморфизм
\end{itemize}

\subsection{Функциональное программирование}

Функциональное программирование накладывает ограничение на
присваивание. Если мы можем сделать элемент неизменяемым, то
должны сделать его таковым
